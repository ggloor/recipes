\documentclass[twocolumn]{article}
\usepackage{lmodern}
\usepackage{amssymb,amsmath}
\usepackage{ifxetex,ifluatex}
\usepackage{fixltx2e} % provides \textsubscript
\ifnum 0\ifxetex 1\fi\ifluatex 1\fi=0 % if pdftex
  \usepackage[T1]{fontenc}
  \usepackage[utf8]{inputenc}
\else % if luatex or xelatex
  \ifxetex
    \usepackage{mathspec}
  \else
    \usepackage{fontspec}
  \fi
  \defaultfontfeatures{Ligatures=TeX,Scale=MatchLowercase}
\fi
% use upquote if available, for straight quotes in verbatim environments
\IfFileExists{upquote.sty}{\usepackage{upquote}}{}
% use microtype if available
\IfFileExists{microtype.sty}{%
\usepackage{microtype}
\UseMicrotypeSet[protrusion]{basicmath} % disable protrusion for tt fonts
}{}
\usepackage[margin=2cm]{geometry}
\usepackage{hyperref}
\hypersetup{unicode=true,
            pdftitle={terrine},
            pdfauthor={Greg Gloor},
            pdfborder={0 0 0},
            breaklinks=true}
\urlstyle{same}  % don't use monospace font for urls
\usepackage{graphicx,grffile}
\makeatletter
\def\maxwidth{\ifdim\Gin@nat@width>\linewidth\linewidth\else\Gin@nat@width\fi}
\def\maxheight{\ifdim\Gin@nat@height>\textheight\textheight\else\Gin@nat@height\fi}
\makeatother
% Scale images if necessary, so that they will not overflow the page
% margins by default, and it is still possible to overwrite the defaults
% using explicit options in \includegraphics[width, height, ...]{}
\setkeys{Gin}{width=\maxwidth,height=\maxheight,keepaspectratio}
\IfFileExists{parskip.sty}{%
\usepackage{parskip}
}{% else
\setlength{\parindent}{0pt}
\setlength{\parskip}{6pt plus 2pt minus 1pt}
}
\setlength{\emergencystretch}{3em}  % prevent overfull lines
\providecommand{\tightlist}{%
  \setlength{\itemsep}{0pt}\setlength{\parskip}{0pt}}
\setcounter{secnumdepth}{0}
% Redefines (sub)paragraphs to behave more like sections
\ifx\paragraph\undefined\else
\let\oldparagraph\paragraph
\renewcommand{\paragraph}[1]{\oldparagraph{#1}\mbox{}}
\fi
\ifx\subparagraph\undefined\else
\let\oldsubparagraph\subparagraph
\renewcommand{\subparagraph}[1]{\oldsubparagraph{#1}\mbox{}}
\fi

%%% Use protect on footnotes to avoid problems with footnotes in titles
\let\rmarkdownfootnote\footnote%
\def\footnote{\protect\rmarkdownfootnote}

%%% Change title format to be more compact
\usepackage{titling}

% Create subtitle command for use in maketitle
\newcommand{\subtitle}[1]{
  \posttitle{
    \begin{center}\large#1\end{center}
    }
}

\setlength{\droptitle}{-2em}

  \title{terrine}
    \pretitle{\vspace{\droptitle}\centering\huge}
  \posttitle{\par}
    \author{Greg Gloor}
    \preauthor{\centering\large\emph}
  \postauthor{\par}
      \predate{\centering\large\emph}
  \postdate{\par}
    \date{09 December, 2018}

\usepackage{geometry}
\usepackage{amsmath}
\newcommand{\ith}[1]{ #1\textsuperscript{th}\ }
\newcommand{\vect}[1]{\vec{\textbf{#1}}}
\setlength{\columnsep}{18pt} 

\setlength\textwidth{5.5in}
\setlength\marginparwidth{1.5in}

\begin{document}
\maketitle

{
\setcounter{tocdepth}{2}
\tableofcontents
}
\hypertarget{terrine}{%
\section{terrine}\label{terrine}}

\textbf{Mike Cyze Sr.~made this, or a variant, every Christmas Eve
dinner. It was always a highlight!}

Hi Greg!

As requested, here is the recipe that you requested\ldots{} It is
intended for one full and one 3/4 full terrines, oval, each about 9x6
inches, about 3 1/4 tall, with lid on top. I have several and you are
free to borrow them. As an aside, do not use a square or rectangular
terrine, as the p\(\^a\)t\(\'e\) will not cook well.

The stuffing is made in several stages! It takes about one morning to
prepare everything and about 1 1/2 hours to cook.

\hypertarget{stuffing}{%
\paragraph{Stuffing :}\label{stuffing}}

\begin{enumerate}
\def\labelenumi{\arabic{enumi})}
\item
  1/2 cup finely minced cooking onion
\item
  2 TB butter
\end{enumerate}

\begin{itemize}
\tightlist
\item
  Note: Cook onion slowly, for about 8 minutes till translucent.
\end{itemize}

Scrape into a large mixing bowl\ldots{}.

1/2 cup port or Madeira --- according to your taste, reduce about 40\%
in skillet

Pour into mixing bowl.

\begin{enumerate}
\def\labelenumi{\arabic{enumi})}
\setcounter{enumi}{2}
\tightlist
\item
  1 lb ground pork
\item
  1 lb ground beef
\item
  3/4 lb ground turkey
\item
  3 eggs, lightly beaten
\item
  2 tsp salt
\item
  1/4 tsp black pepper
\item
  Big pinch of allspice
\item
  1/2 tsp thyme
\end{enumerate}

Add all the ingredients to mixing bowl and beat vigorously with a wooden
spoon till the mixture has lightened in texture and is thoroughly
blended. This requires arm strength!

Saut\(\'e\) a small mouthful to check for salt, pepper, etc. It must be
perfectly flavoured!

Set the stuffing aside.

\hypertarget{marinate-the-duck-can-use-veal-or-ham-if-desired-and-not-too-fussy}{%
\subsubsection{Marinate the duck (can use veal or ham if desired and not
too
fussy)}\label{marinate-the-duck-can-use-veal-or-ham-if-desired-and-not-too-fussy}}

In a bowl place:

\begin{enumerate}
\def\labelenumi{\arabic{enumi})}
\item
  Two small skinless magrets sliced into 1/2 inch strips
\item
  3 or 4 TB cognac or Armagnac or good brandy
\item
  Big pinch of salt and pepper
\item
  Pinch of thyme
\item
  Pinch of allspice
\item
  2 TB finely minced shallots or green onion
\end{enumerate}

Marinate the duck in the bowl with the cognac and seasonings. Set aside.

Before using separate the meat from the marinade. Reserve the marinade.

\hypertarget{putting-it-all-together}{%
\subsection{PUTTING IT ALL TOGETHER}\label{putting-it-all-together}}

Preheat oven to 350 deg

8 cup oval terrine (as noted above), second terrine.

Sheets or strips of fresh pork fat back or bellies, pounded or sliced to
1/8 inch thick

Duck stuffing -above-

Bay leaf

Sheet of pork fat or strips to cover terrines

\begin{itemize}
\tightlist
\item
  Line terrines with sheets or strips of pork fat, including sides of
  terrines.
\end{itemize}

Beat reserved marinade into pork/beef/turkey stuffing (above)

Dip hand in cold water so meat does not stick to hand

Place about 1 to 1 1/2 inch of stuffing onto pork fat

Arrange some duck strips onto stuffing, unevenly

Cover with more stuffing, perhaps 3/4 inch thick, pack down

Arrange more duck strips, not close to edge

Add more stuffing, more or less level with top of terrine

Place bay leaf on top

Cover with sheet or strips of pork fat

Enclose top of terrine with aluminium foil\ldots{}

Cover with lid.

Repeat for second terrine with left overs.

May be only 3/4 full\ldots{}

Place the terrines in a pan of boiling water (I use a bottom of roaster)

So that the water comes up about halfway on side of terrines

Add water if it gets too low

Set in lower third of preheated oven and cook for 1 1/2 hours.

The pâté is done when it has shrunk slightly and the juices run clear.

Now the critical part.

Take the terrines out of the water and set them on a board or old
towels.

Remove lid and place an oval piece of wood on top of terrine. (I have
some).

On it place a 3 to 4 lbs weight so as to pack the terrine and there are
no air holes left. (I use a brick). The clear juices will run onto
towels.

Allow to cool at room temperature for a few hours and then chill it
overnight.

AND THEN TRY NOT TO EAT TOO MUCH OF IT\ldots{} (after removing the foil
and bay leaf)

Basically this is it.. ! In past years I ground my own meat and even
carved fresh duck: that was a waste of time! Grocers carry your
requirements!

HAVE FUN! m


\end{document}
